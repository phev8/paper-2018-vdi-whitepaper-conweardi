\section{Introduction}
For digitization projects, the construction industry is a mostly unconquered field yet.
On the one hand, the dynamic, sometimes harsh (e.g. weather influence, no infrastructure, etc.) environments of the construction site challenges the technology and adds harder requirements compared to an industrial application on the shop floor. 
On the other hand, the typically conservative and hierarchical work culture makes the introduction and acceptance of new methods more challenging. 
In this paper, we describe the vision and ideas behind the ConWearDi project and its approach to better understand and overcome these challenges.

\section{Project scope}
Construction projects can vary in size and complexity...
In our project, we focus on smaller constructions like house construction (compared to large civil engineering projects), where the work on one site is accomplished mostly by few on-site workers. Nonetheless, managing workflows and resources for such craftsman businesses can be very challenging.

\section{Problem description}
The construction industry shows a huge potential for digitization and it's still in the early phase of adopting technologies related to Industry 4.0. 
Meanwhile the Building Information Modeling (BIM) supports the processes of the design and planing phase, the actual construction, the execution process of the value creation, is still dominated by analog processes and paper documents. 
Examples include wall sized printed plans and time sheets on paper, which are only digitally recorded in the office and are available only there. 
So, in many cases, the digital world ends in the back office or at the workstations of the architects and engineers. 
The potential of novel services, such as Internet of Things (IoT) systems with sensors and actuators deployed on site connecting it to powerful computing resources, remained untapped.

\section{Project goal}
Our main goal in the ConWearDi project is the design and prototype implementation of a platform capable to capture and analyze the current state of the construction work and to provide useful informational support not only for the managers and engineers but directly to the construction workers and craftsmen on site. 

%Information sources, which provide structured information and accurate specifications (e.g. used materials) about a project, such as the BIM itself, are the foundation to achieve this goal. These

The main project goals can be summarized as:
\begin{enumerate}
  \item Creation of a Digital Twin, a representation of the construction site, which always reflects its current state 
  \item Centralized system for managing the construction site
  \item Automatic documentation for knowledge- and quality management purposes
  \item Exploration of new business models
  \item Development of methods for monitoring the activities of different worker groups with wearable technologies
\end{enumerate}

\section{Methodology}
\todo{to succeed with the complex task - include experts from different professions and stakeholders}
\todo{figure describes the main functional process of the project - ... describe here }

\begin{figure*}
\includegraphics[width=0.8\textwidth]{figures/conweardi-functional}
\caption{Functional graph}
\end{figure*}

\todo{project partners, who is doing what}

\todo{add paper references here to previous works of the groups}

\section{Conclusion}

\todo{describe user benefits, when project realized (last part of Michaels doc)}

\todo{for more information visit project website}


\begin{acks}
  \todo{BMBF founded project}
  
  The work is
  supported by the \grantsponsor{GS501100001809}{National Natural
    Science Foundation of
    China}{http://dx.doi.org/10.13039/501100001809} under Grant
  No.:~\grantnum{GS501100001809}{61273304}
  and~\grantnum[http://www.nnsf.cn/youngscientists]{GS501100001809}{Young
    Scientists' Support Program}.

\end{acks}
